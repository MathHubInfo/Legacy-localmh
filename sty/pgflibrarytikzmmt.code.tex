%%%%%%%%%%%%%%%%%%%%%%%%%%%%%%%%%%%%%%%%%%%%%%%%%%%%%%%%%%%%%%%%%
% A TIKZ library for MMT Theory Graphs
% copyright 2014 Michael Kohlhase; Released under the LPPL
%%%%%%%%%%%%%%%%%%%%%%%%%%%%%%%%%%%%%%%%%%%%%%%%%%%%%%%%%%%%%%%%%
%
% this library provides some standardized node and arrow styles for formatting MMT graph
% diagrams in tikz. The advantage is that we can classify the arrows and nodes
% symbolically and with the styles in this library achieve a uniform look that helps
% readability. 
%
%%%%%%%%%%%%%%%%%%%%%%%%%%%%%%%%%%%%%%%%%%%%%%%%%%%%%%%%%%%%%%%%%

% 1. Arrows 
% 1.1. Arrowtips (only internal)
\usetikzlibrary{arrows}
\newcommand{\@mmtarrowtip}{angle 45}
\newcommand{\@mmtreversearrowtip}{angle 45 reversed}
\newcommand{\@mmtarrowtipepi}{triangle 45}
\newcommand{\@mmtarrowtipmonoright}{right hook}
\newcommand{\@mmtarrowtipmonoleft}{left hook}
\newcommand{\@mmtarrowtippartial}{right to}
\newcommand{\@mmtarrowtippartialleft}{left to}
\newcommand{\@mmtreversearrowtippartial}{right to reversed}
\newcommand{\@mmtreversearrowtippartialleft}{left to reversed}

% 1.2 the arrow sstyles in graphs
%
% a generic theory
\tikzstyle{thy}=[draw,outer sep=.6mm,rounded corners]
% a flat theory
\tikzstyle{flatthy}=[thy,double]
% a theory graph 
\tikzstyle{thygraph}=[draw,outer sep=1mm,rounded corners,dashed]

% any morphism 
\tikzstyle{morph}=[-\@mmtarrowtip] 
\tikzstyle{mapsto}=[|-\@mmtarrowtip] %| any morphism 
% structures
\tikzstyle{struct}=[-\@mmtarrowtip,thick]
% inclusions: regular, partial, and left variants
\tikzstyle{include}=[\@mmtarrowtipmonoright-\@mmtarrowtip,thick]
\tikzstyle{pinclude}=[\@mmtarrowtipmonoright-\@mmtarrowtippartial,thick]
\tikzstyle{includeleft}=[\@mmtarrowtipmonoleft-\@mmtarrowtip,thick]
\tikzstyle{pincludeleft}=[\@mmtarrowtipmonoleft-\@mmtarrowtippartialleft,thick]
% views: regular, mono, partial, and left variants
\tikzstyle{view}=[dashed,thick,-\@mmtarrowtip]
\tikzstyle{mview}=[dashed,thick,\@mmtarrowtipmonoright-\@mmtarrowtip]
\tikzstyle{pview}=[dashed,thick,-\@mmtarrowtippartial]
\tikzstyle{pmview}=[dashed,thick,\@mmtarrowtipmonoright-\@mmtarrowtippartial]
\tikzstyle{viewleft}=[dashed,thick,-\@mmtarrowtip]
\tikzstyle{mviewleft}=[dashed,thick,\@mmtarrowtipmonoleft-\@mmtarrowtipleft]
\tikzstyle{pviewleft}=[dashed,thick,-\@mmtarrowtippartialleft]
\tikzstyle{pmviewleft}=[dashed,thick,\@mmtarrowtipmonoleft-\@mmtarrowtippartialleft]
% biviews: regular, partial, and left variants
\tikzstyle{biview}=[dashed,thick,\@mmtreversearrowtip-\@mmtarrowtip]
\tikzstyle{pbiview}=[dashed,thick,\@mmtreversearrowtippartial-\@mmtarrowtippartial]
\tikzstyle{pbiviewleft}=[dashed,thick,\@mmtreversearrowtippartialleft-\@mmtarrowtippartialleft]
% defining views (experimental)
\tikzstyle{defview}=[dashed,thick,double,-\@mmtarrowtip]
% meta-theory inclusion
\tikzstyle{meta}=[dotted,-\@mmtarrowtip,thick]
% conservative extensions (abbreviation)
\tikzstyle{conservative}=[hooks-\@mmtarrowtip,double] 

% 2. convenience macros
% 2.1 the \mmtthy macro takes three arguments, name, decl, axioms and makes a 
% table-like structure
\newcommand\mmtthy[3]{\def\@test{#3}%
\begin{array}{l}\textsf{#1}\\\hline #2\ifx\@test\@empty\else\\\hline #3\fi\end{array}}
% 2.2 the \mmtar takes two arguments, some tikz options, and an arrow style. \nmmtar
% is a variant that also has a name on top.
\newcommand\mmtar[2][]{\raisebox{.5ex}{\tikz[#1]{\draw[#2] (0,0) -- (.6,0);}}}
\newcommand\nmmtar[3][]{\raisebox{.4ex}{\tikz[#1]{\draw[#2] (0,0) --
      node[above]{\ensuremath{\scriptstyle #3}} (.8,0);}}}
